\documentclass{jsarticle}

\newcommand{\T}{\mathsf{T}}

\begin{document}
\section{中心化}
$n$個の対象に対する特徴量を$p$とした時,$n \times p$を$\bm X = x_0,\udots,x_n$と定義する.
また距離行列$D^2$の要素$d_{ij}$を次のように定義する.
\begin{eqnarray}
    d_{ij}=\| x_i - x_j\|^2
\end{eqnarray}
また,距離行列$D^2$は$\bm X$を用いると次のように表すことができる.
\begin{eqnarray}
    D^2 = diag(X^\T X)1_n1_n^T - 2X^\T X + 1_n1_n^T diag(X^\T X)
\end{eqnarray}
diang($X X^T$)は対角成分を取り出した行列である.
また,中心化行列は次のように表される.
\begin{eqnarray}
    J_n =   I_n - \frac{1}{n}1_n 1_n^\T
\end{eqnarray}
\end{document}